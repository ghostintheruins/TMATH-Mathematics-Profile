\documentclass{article}


%\usepackage{showkeys}

\usepackage{amssymb}
\usepackage{amsmath}
\usepackage{hyperref}
\usepackage{color}
\usepackage{graphicx}
\usepackage{placeins}



\title{Section 15.3 Structure}
\author{Rain Wilson}
\date{September 2020}

\begin{document}
\maketitle

\begin{enumerate}

    \item Reminds the reader that we talked about both equidecomposability and equicomplementability in Chapter 3 in cases of two-dimensional objects (p.382).
    
    \item Includes definitions of both concepts so that the reader has an idea of what these terms mean when they are demonstrated further in the paper (p.382).
    
    \item Asks Kyuta, but also (implied) the reader to think about these concepts, and if they can be applied or are relevant to three-dimensional objects (p.382).
    
    \item Provides context through a brief history on the origin of the problem (p.382).
    
    \item Defines the problem itself (p.382).
    
    \item States that solutions were found, who found them, and when and how the proofs of the solutions to the original problem were presented to the public (p.382).
    
    \item Provides more context in a sentence about an improvement to Dehn's proofs by Hadwiger, Kagan, and Boltyanskii (p.383).
    
    \item States in a short paragraph the things they're going to be doing to demonstrate the Dehn Invariant, Dehn's theorem and solution, and Hadwiger's expansion on Dehn's work on the topic (p.383).
    
    \item Uses general mathematical equations to define the Dehn Invariant (p.383).
    
    \item Takes a parallelpiped and uses it as an example to demonstrate the properties of the Dehn Invariant and how it works (p.383).
    
    \item Provides an explanation for why all parallelohedra share the same Dehn Invariant of zero (0) by listing the properties that all parallelohedra share that are relevant to it (p.384).
    
    \item Provides a picture of a parallelohedron so that the reader might be able to understand how the properties and general equations apply (p.384).
    
    \item (I'm really not sure about this one. It's a little confusing to me.) Defines Dehn's Lemma whose general equations provide a a way to calculate the Dehn Invariant of polyhedra to show the reader that equidecomposability and equicomplementability are not always possible in three dimensions (p.384).
    
    \item Asks Kyuta (also, the reader) to calculate the Dehn Invariant of a specific polyhedron, and explains how that polyhedra is is one of three congruent square base pyramids inside a regular cube so that the reader can visualize the problem and to aid in the process of finding a solution to the proposed question (p.384)
    
    \item Demonstrates by calculation of the general equations, the equalities that are present in the specified polyhedra decomposed from the cube, by Dehn's Lemma (pp.384-385).
    
    \item Provides a diagram of the object in question, giving context to it by also including diagrams of the cube from which it was contained, all of the cuts inside the cube, and the three congruent objects cut from the cube in their original rotational positions outside of the cube (pp.384-385).
    
    \item Gives a brief confirmation of the previous solution (p.385).
    
    \item Gives an overview of the examples they are going to use involving polyhedra obtained through decomposing a regular cube to provide proof of Dehn's Lemma, and why the previous solution is true (p.385).
    
    \item Provides a diagram of the original cube containing all of the cuts and relevant labels that are going to be used in all three examples (p.385).
    
    \item Goes through the calculations and explanations of all three examples (pp.385-386).
    
    \item (I think they could use a recall note here, because with all those examples, definitions, and diagrams, I've forgotten that we talked about those dead guy's improvements somewhere back at the beginning of the section.. or is that just my brain injury?) Uses another small outline to explain that they are going to use Dehn's Lemma to proof the theorem that better explains Dehn's proofs (p.386).
    
    \item Defines the Dehn-Hadwiger theorem (p.386).
    
    \item Provides proof by contradiction of the aforementioned theorem using calculations of general equations (p.386).
    
    \item Explains that the process of cutting up two parallelohedra and composing each into identical new polyhedra if reversible, should have the same Dehn Invariant value of zero (p.387).
    
    \item Gives another single case outline of the next example they are going to use (p.387).
    
    \item (I could use a recall note here, too.) Provides diagrams of two polyhedra inscribed in cubes of the same volume as a visual introduction to the example for the original problem stated at the beginning of the section (p.387).
    
    \item Explains the two polyhedra represented in the previous diagrams, and includes another diagram of the first inscribed polyhedron outside of it's cube, giving it numerical dimensions for the next calculations (p.388).
    
    \item Proceeds with the calculations using the previously given values, thereby completing the demonstration of the original problem (p.388).
    
\end{enumerate}

I probably shouldn't have been so detailed. I'm really not good at this brevity stuff.

\end{document}