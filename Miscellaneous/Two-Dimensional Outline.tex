\documentclass{article}


%\usepackage{showkeys}

\usepackage{amssymb}
\usepackage{amsmath}
\usepackage{hyperref}
\usepackage{color}
\usepackage{graphicx}
\usepackage{placeins}



\title{Two-Dimensional Outline}
\author{Rain Wilson}
\date{September 2020}

\begin{document}
\maketitle

\paragraph{} I think we might be able to use the structural outline I just did for 15.3 as a baseline for our own paper.

\begin{enumerate}

    \item Give some context and a brief history the problem and proofs we are going to demonstrate in a short introduction.
    
    \item Include definitions of equidecomposability and equicomplementability, so that the reader has an idea of what these terms mean when they are demonstrated further in the paper.
    
    \item Define the problem itself.
    
    \item State that solutions to the problem were found, who found them, and when and how the proofs of the solutions to the original problem were presented to the public.
    
    \item Outline, in a short paragraph, the things we are going to be doing to demonstrate the original problem in regards to the concept of equidecomposability, the solutions and their proofs, and the relevant theorem(s).
    
    \item Use general mathematical equations to define whatever we're trying to do in the next examples.
    
    \item Devise and use a basic example of polygons to demonstrate the properties.
    
    \item Provide an explanation for why equidecomposability is possible for a pair any two polygons.
    
    \item Provide a visual representation of the polygons we are going to use in our example so that the reader might be able to understand how the properties and general equations apply.
    
    \item Go through the example calculations and explain its conclusion.
    
    \item Ask the reader to consider another example of a specific polygons, and use the previously mentioned equations to calculate applicable information.
    
    \item Explain how the previously specified polygons are obtained and provide diagrams showing the cuts that are used.
    
    \item Demonstrate by calculation of the general equations the properties that are present in the specified polygons.
    
    \item Give a brief confirmation of the previous solution.
    
    \item Give an overview of the next example we are going to use that explains why our previous solutions hold true.
    
    \item Provide a diagrams for said next example.
    
    \item Go through the general calculations and explanations required for the demonstration.
    
    \item Explain what happened in those calculations and why they support whatever it was that I forgot that we were trying to do.
    
    \item Provide a contextual recall note for the next part of our demonstration by using another brief outline.
    
    \item Outline, in a short paragraph, the things we are going to be doing to demonstrate the original problem in regards to the concept of equicomplementability, the solutions and their proofs, and the relevant theorem(s).
    
    \item Provide proof of the aforementioned theorem using calculations of general equations.
    
    \item Explain that the process of cutting up two polygons and composing each into perfectly superimposed new polygons.
    
    \item Gives another outline of the next example(s).
    
    \item Perhaps provide another recall note.
    
    \item Describe the polygons we are going to use in the next example(s) and provide diagrams of them.
    
    \item Explains the two polygons represented in the previous diagrams, and give them numerical values for exact computation.
    
    \item Proceed with the calculations using the previously given values, thereby completing the demonstration of equicomplementability and the original problem.
    
\end{enumerate}

I tried to be more general. I don't know if I succeeded, though. I also don't know if I got everything. We can revise this later, though.

\end{document}